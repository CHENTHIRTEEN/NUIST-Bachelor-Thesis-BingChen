\pagenumbering{arabic}
\section{绪言}
\subsection{研究背景和意义}
公共交通,泛指所有向大众开放、并提供运输服务的运输系统,
该系统通常按照固定的时刻表进行资源调度管理,在既定的线路上运行,
每次向乘客收取一定的费用。公交系统作为城市公交网络的重要组成部分,
其发展水平直接影响了市民出行的便捷性和幸福感,
对建设和谐社会及和谐城市有着重大意义。
以南京市公共交通系统为例,2022年第一季度,全市营业性客车达6353辆,客位279619座,
全市客运班车总数达879辆、45428客位,旅游及包车5474辆、240286客位。
普通干线公路日均交通量29.52万辆\cite{njjt}。


自国家十四五规划提出“智慧交通建设”以来,
各种交通方式发展从传统要素驱动转向创新科技驱动。
根据高德地图发布的《2022年度中国主要城市交通分析报告\cite{x1}》,
2019至2022年度大中型城市的整体候车时长同比呈上升趋势,
尤其是受发车频率影响的候车时长上升明显;主要城市高峰平均交通拥堵指数达到1.704
(在通行一小时的行程中实际花费为1.704小时)。

城市人口的快速增长造成了交通拥堵,同时交通拥堵也阻碍了城市经济的发展。
倡导公交出行是解决交通拥堵的一个主流方案。
公交系统的发展与居民生活息息相关,
根据百度地图发布的《2022年度中国城市交通报告\cite{bd}》,
我国主要城市公交候车时间为8.97$ \sim $17.34分钟,
且候车时间越长的城市受发车频率和交通扰动影响也越大。
如何构建一个高效稳定的交通系统,提高乘客的出行满意度,
以吸引更多乘客选择公共交通出行,对推动绿色出行和社会发展有着重大意义。
随着数据挖掘、机器学习等

本文以南京信息工程大学内的小型公交系统为例,对公交的刷卡数据、线路数据、
天气数据等进行处理,通过大数据分析和机器学习算法对乘客的行为进行挖掘,
优化站点设置和发车策略,提高乘客出行质量,增强公共交通的竞争力。

\subsection{本模板介绍}
鉴于Microsoft Word 编写论文排版工作量大且枯燥,为了实现论文的排版的自动化、规范化,按照《南京信息工程大学本科生毕业论文(设计)撰写排版规范》编写了可用于南京信息工程大学本科生毕业论文的\LaTeX 模板。

模板2021.6版本修订者根据2021年的论文格式要求修订该模板,并使用该模板完成毕业论文工作。