
\mytableofcontents

\maketitleofchinese{公共交通调度策略的最优化方法\\——以南京信息工程大学校园为例}{陈冰}{应用技术}

\abstractofchinese{目标规划是运筹学中的一个重要分支。在一些复杂场景下的目标规划问题可能不光要满足一个单一的目标。又称多目标最优化。随着目标约束的增多,模型复杂度也更高。本文基于启发式双目标规划算法,给出一种满足乘客较小等待时间和较小空车率的多目标规划算法。使用Pyomo对该问题进行建模,通过数值求解器Gurobi对该
算法进行多次迭代求解,得到新的时刻表。通过实地统计的真实数据,使用离散事件模拟包SimPy对校园公交系统内的排队乘客进行模拟。通过实验证
明,在相同量级数量的参数约束下,该启发式目标约束算法有效降低排队乘客的平均等待时间。}{运筹优化;多目标规划;公交调度;启发式方法}

\maketitleofenglish{The Optimization Method of Public Transport Scheduling Strategy\\——A Case Study of Nanjing University of Information Science and Technology
}{Bing Chen}{School of Applied Technology}

\abstractofenglish{Objective planning is an important branch in operations research. In some complex scenarios, the goal planning problem may be more than just meeting a single goal. Also known as multi-objective optimization. As the target constraints increase, so does the complexity of the model. Based on the heuristic dual-objective planning algorithm, this paper presents a multi-objective planning algorithm that satisfies the small waiting time and small empty rate of passengers. The problem was modeled using Pyomo, which was applied using the numerical solver Gurobi
The algorithm solves several iterations to obtain a new timetable. Using real-world data from field statistics, SimPy was used to simulate queued passengers in the campus bus system. Pass real-world verification
Obviously, under the same number of parameter constraints, the heuristic target constraint algorithm effectively reduces the average waiting time of queuing passengers.}
{Operations research optimization; Multi-objective planning; Bus scheduling; Heuristics}
