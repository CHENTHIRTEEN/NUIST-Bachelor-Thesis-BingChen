\section{总结与展望}
本文基于传统的启发式方法,使用一种包含多个约束的目标规划算法,该算法灵感来源于
\cite{CHIERICI200499,2002},对传统的有规则时刻表计算任务添加多个目标约束,
同时由 Gurobi Optimization 公司开发新一代大规模优化器Gurobi,为该类问题的
建模和求解提供了帮助。当实际问题越来越复杂,问题规模越来越庞大的时候,需要一个经过
证明可以信赖的大规模优化工具,为决策提供质量保证。基于3.1中经典的排队理论,基于真实的
统计数据,可以使用离散系统建模包SimPy对车站客流进行模拟。
本文重点讨论了如何通过计算和调整具有给定时间变化的乘客的时刻表,使乘客在车站的总等待时间最小。
建立了一个统一的带有线性约束的二次整数规划模型,来协调每个车站的乘客出发时间。
如图~\ref{fig34}~到图~\ref{fig36}~的模拟结果,通过平均队列长度、平均等待时间等指标,
分析现有系统的缺陷,给出适当的优化方案。通过多次迭代解得出的新的发车表(图~\ref{fig45}~)
在理论上可以有效提高系统的平均服务强度。

同时本文的优化问题具有一定的普遍性,在大数据时代,机器学习、深度学习技术热火朝天的同时,
优化问题再一次进入人们的视野,“十年电气无人问,一朝ML人皆知”,这句话可见现有新研究对优化问题的重视。
随着AI技术的发展,对机器学习底层的最优化理论的重要性已经开始逐渐意识到了。
最优化理论是机器学习算法的核心,包括梯度下降、牛顿法、拟牛顿法等一系列优化算法,
这些算法的效率和收敛性直接影响到机器学习的性能和应用。理解最优化理论的基本原理和方法,
对于开发高效、鲁棒的机器学习算法至关重要。越来越多的人开始关注最优化理论,并且开始进行相关的研究和开发。
最优化以蕴含在机器学习内部,作为参数寻优的调整方向。大可以作为独立模型,在机器学习得到的数据模型的基础上做决策模型,
比如现在的随机优化已有先拟合分布再建立优化的方向。

本文同样还存在许多的不足,由于数据采集条件有限,对于客流分布的描述仅仅基于少量的统计数据得出,
进行计算后给出客流的分布函数,对于乘客出行兴趣、O-D对等数据没有进行详尽的统计。
事实上客流受非常多的因素影响,仅仅用简单的分布函数难以完整的描述出该模式。
由于道路网络时变的交通模式和复杂的空间依赖性,交通流预测是一个具有挑战性的时空预测问题。
为了克服该挑战,信德烟酒将交通网络看为一张图,提出新的深度学习预测模型,
交通图卷积长短时记忆网络(TGC-LSTM)学习交通网络中道路之间的相互作用,并预测网络级的交通状态。
近年来的一些研究\cite{bjtu2020}也表明,传统的LSTM网络在对于融合乘客出行兴趣、O-D对、天气等因素的复杂场景下,
对于客流的预测有着不俗的表现。对于更加复杂的场景,融合时空的双向数据下,基于self-attention的算法\cite{transformer2022}
更是有着优于LSTM的表现。现有的交通流预测方法大部分采用基于区域的态势感知图像或基于站点的图像表示去捕捉交通流的空间动态性,
而态势感知图像和图形表示的结合同样是精准预测的关键。
在对于优化模型的的求解,本文提出的模型在迭代次数超过一定程度后,不具有良好的
收敛性。对于大规模数值优化问题的求解,基于深度学习的方法\cite{nair2021solving}近年来也进入了人们的视野。显然,当前大数据时代,
基于数据驱动的机器学习方法是大势所趋,传统优化方法与新技术的结合受到越来越多的重视。期待在有了更加优质的数据的支持下,
可以使用新的方法对该问题进行更深入的研究。



