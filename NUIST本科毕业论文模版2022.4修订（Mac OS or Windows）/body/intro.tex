\pagenumbering{arabic}
\section{绪言}
\subsection{研究背景和意义}
公共交通,泛指所有向大众开放、并提供运输服务的运输系统,
该系统通常按照固定的时刻表进行资源调度管理,在既定的线路上运行,
每次向乘客收取一定的费用。广义而言,公共交通包括民航、
铁路、公路、水运等交通方式。公交系统作为城市公交网络的重要组成部分,
其发展水平直接影响了市民出行的便捷性和幸福感,
对建设和谐社会及和谐城市有着重大意义。公交系统与居民的日常出行息息相关,
是居民日常出行的主要选择。
以南京市公共交通系统为例,2022年第一季度,全市营业性客车达6353辆,客位279619座,
全市客运班车总数达879辆、45428客位,旅游及包车5474辆、240286客位。
普通干线公路日均交通量29.52万辆\cite{njjt}。但是城市人口的快速增长造成了交通拥堵,同时交通拥堵也阻碍了城市经济的发展。
根据高德地图发布的《2022年度中国主要城市交通分析报告\cite{x1}》,
2019至2022年度大中型城市的整体候车时长同比呈上升趋势,
尤其是受发车频率影响的候车时长上升明显;主要城市高峰平均交通拥堵指数达到1.704
(在通行一小时的行程中实际花费为1.704小时)。
根据百度地图发布的《2022年度中国城市交通报告\cite{bd}》,
我国主要城市公交候车时间为8.97$ \sim $17.34分钟,
且候车时间越长的城市受发车频率和交通扰动影响也越大。
倡导公交出行是解决交通拥堵的一个主流方案,
城市公交系统对节能减排,建设绿色城市也有着重要意义,公交系统依赖城市道路网络,且大多为电力驱动,
更多的载客量下产生的碳排放更少,且良好的站点设置与调度方式可以让公交系统与共享单车、
轨道交通等其他出行方式结合产生完美的绿色出行生态系统。如何构建一个高效稳定的交通系统,
提高乘客的出行满意度,以吸引更多乘客选择公共交通出行,对推动绿色出行和社会发展有着重大意义。


交通调度优化是一项十分复杂的系统性工程,多种因素相互合作用影响着整个系统。
随着大数据技术的发展,各行各业每天产生海量的数据,越来越多的特征被加入模型中,
海量数据的支持让机器学习在各行各业崭露头角。使用机器学习与数据挖掘算法,通过对道路、天气、乘客等数据的
挖掘分析,可以更准确的建立乘客出行模式的模型。在建立了乘客出行模式的模型后,可以对客流量进行更好的预测。
传统的运筹学方法对公交调度这类有约束资源分配问题下有良好的表现,但是在大数据时代难以发挥作用,
将机器学习与运筹学结合,使用机器学习算法更好的对不同场景下的乘客出行喜好进行分析、对客流量进行预测。
通过所得的数据建立相应问题的规划模型并求解,更好的解决相应交通规划问题。


本文以南信大校园公交系统为例,探究在大数据时代下,如何结合传统运筹优化算法与机器学习的优势,
对公交系统进行优化。现有运营方案下,校园公交人流量高峰时期常采用多车次连续发车的方法,
由于公交线路站点大多均匀坐落在校园主干道上,高峰时期车辆行进速度也会受到主干道交通环境影响,
导致部分站点排队时间较长;非高峰时期,为考虑运营成本,公交采用固定弹性时刻表方式进行调度,
现有的调度策略存在滞后性。通过对客流大数据进行挖掘分析,
运用机器学习算法对客流量模式进行识别,通过该模式建立数学模型,使用运筹学方法从成本、效益、
乘客舒适度等多个角度出发,对公交站定址、调度策略等进行优化,
最终解决现有策略下校园公共交通系统中的缺陷。


\subsection{研究现状}
常规情况下,公共交通调度采用有规则定期时刻表的方式进行调度,2004年,Alessandro Chierici
等人扩展了此种普遍被采用方法\cite{c1},新方法考虑时间表的质量与交通系统对其他可替代的运输方式吸引的
乘客之间的相互影响。在放宽完整性约束后,所得的非线性混合整数模型仍然是非
凸的。通过基于外部近似的分支定界算法和利用两个子模型的分解和相互更新
的启发式算法来解决它。随着机器学习近十年的飞速发展带动了各行各业对于数据驱动的研究热潮,
运筹学也不例外,2020年Google与Deepmind提出一种基于深度神经网络提升传统混合整数规划方法的性能\cite{nair2021solving}。
传统的混合整数规划中通常是基于分支定界法(branch and bound),
通过不断求解连续凸松弛问题找出最优解,但是理论上需要求解的松弛问题个数随着整数
变量维度上升呈指数倍数增长。因此,实际中会添加许多加速方法去减少需要求解的
松弛问题数量,其中有很多加速方法的效果取决于混合整数规划的问题结构或是
当前分支定界已有的信息,但是很多时候由于模型过于复杂,可能无法直接提取出这类信息,
而这正好是深度学习方法的优势,
它能基于充足的数据通过黑盒优化提供一个预测模型,捕捉一些难以解释的有效信息。
核心是使用神经网络提升传统混合整数规划中两种常见的启发式方法的性能:
下潜(Diving)搜索和分支(branching)选取。
其主要思想还是利用神经网络基于数据的预测能力对某一类特定问题求解实现加速。

\subsection{研究内容}

\subsection{论文组织结构}
本文拟分为五章来对该课题进行研究:

第一章主要介绍研究背景、研究意义、国内外研究现状、主要研究内容和结构。

第二章对公交的刷卡数据、线路数据、天气数据等进行处理,
通过大数据分析和机器学习算法对乘客的行为进行挖掘,
本章将介绍设计的相关算法。通过数据挖掘,分析影响乘客行为模式的特征。

第三章介绍运筹优化中公交选址、班次调度的不同优化算法,
对各个算法的效果和特点进行分析。
第四章通过前文的分析结果建立数学模型,
从成本、效益、乘客舒适度等多个角度出发进行数值仿真模拟,
对比不同优化结果的有效性。

第五章总结本文工作、并且展望未来研究方向。