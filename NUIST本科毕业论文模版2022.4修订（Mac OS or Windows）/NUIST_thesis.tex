% !TeX program  = XeLaTeX
% !TeX encoding = UTF-8
\documentclass[a4paper]{nuist}

\begin{document}

%%% 论文封面
% WARN: 请检查校徽是否为最新版,若不是最新版,请在 nuist_logo 文件夹中替换
% TIPS: 若标题、学院、专业名太长,可使用 \\ 进行换行,如 计算机学院、软件学院、\\网络空间安全学院
% TIPS: 若不允许换行,请在 nuist.cls 文件中查找“封面”部分,并将 \parbox[b]{58mm} 中的58改为更大的数字
% TIPS: 2021年论文格式要求指导老师不用加职称
\cover{公共交通调度策略的最优化方法\\——以南京信息工程大学校园为例}
{陈冰}{201933070085}{算机学院、软件学院、\\网络空间安全学院}{软件工程}{马杰良}{二O二三\hspace{0.4em} 年\hspace{0.4em} 五\hspace{0.4em} 月\hspace{0.4em} 七\hspace{0.4em} 日}
%%% 论文封面

%%% 正文部分
% TIPS: 可以为每一章节在 body 文件夹内创建一个 .tex 文件,并以下述方式引入,也可以直接写在本文件中(不推荐)

\mytableofcontents

\maketitleofchinese{公共交通调度策略的最优化方法\\——以南京信息工程大学校园为例}{陈冰}{应用技术}

\abstractofchinese{目标规划是运筹学中的一个重要分支。在一些复杂场景下的目标规划问题可能不光要满足一个单一的目标。又称多目标最优化。随着目标约束的增多,模型复杂度也更高。本文基于启发式双目标规划算法,给出一种满足乘客较小等待时间和较小空车率的多目标规划算法。使用Pyomo对该问题进行建模,通过数值求解器Gurobi对该
算法进行多次迭代求解,得到新的时刻表。通过实地统计的真实数据,使用离散事件模拟包SimPy对校园公交系统内的排队乘客进行模拟。通过实验证
明,在相同量级数量的参数约束下,该启发式目标约束算法有效降低排队乘客的平均等待时间。}{运筹优化;多目标规划;公交调度;启发式方法}

\maketitleofenglish{The Optimization Method of Public Transport Scheduling Strategy\\——A Case Study of Nanjing University of Information Science and Technology
}{Bing Chen}{School of Applied Technology}

\abstractofenglish{Objective planning is an important branch in operations research. In some complex scenarios, the goal planning problem may be more than just meeting a single goal. Also known as multi-objective optimization. As the target constraints increase, so does the complexity of the model. Based on the heuristic dual-objective planning algorithm, this paper presents a multi-objective planning algorithm that satisfies the small waiting time and small empty rate of passengers. The problem was modeled using Pyomo, which was applied using the numerical solver Gurobi
The algorithm solves several iterations to obtain a new timetable. Using real-world data from field statistics, SimPy was used to simulate queued passengers in the campus bus system. Pass real-world verification
Obviously, under the same number of parameter constraints, the heuristic target constraint algorithm effectively reduces the average waiting time of queuing passengers.}
{Operations research optimization; Multi-objective planning; Bus scheduling; Heuristics}

\pagenumbering{arabic}
\section{绪言}
\subsection{研究背景和意义}
公共交通,泛指所有向大众开放、并提供运输服务的运输系统,
该系统通常按照固定的时刻表进行资源调度管理,在既定的线路上运行,
每次向乘客收取一定的费用。公交系统作为城市公交网络的重要组成部分,
其发展水平直接影响了市民出行的便捷性和幸福感,
对建设和谐社会及和谐城市有着重大意义。
以南京市公共交通系统为例,2022年第一季度,全市营业性客车达6353辆,客位279619座,
全市客运班车总数达879辆、45428客位,旅游及包车5474辆、240286客位。
普通干线公路日均交通量29.52万辆\cite{njjt}。


自国家十四五规划提出“智慧交通建设”以来,
各种交通方式发展从传统要素驱动转向创新科技驱动。
根据高德地图发布的《2022年度中国主要城市交通分析报告\cite{x1}》,
2019至2022年度大中型城市的整体候车时长同比呈上升趋势,
尤其是受发车频率影响的候车时长上升明显;主要城市高峰平均交通拥堵指数达到1.704
(在通行一小时的行程中实际花费为1.704小时)。

城市人口的快速增长造成了交通拥堵,同时交通拥堵也阻碍了城市经济的发展。
倡导公交出行是解决交通拥堵的一个主流方案。
公交系统的发展与居民生活息息相关,
根据百度地图发布的《2022年度中国城市交通报告\cite{bd}》,
我国主要城市公交候车时间为8.97$ \sim $17.34分钟,
且候车时间越长的城市受发车频率和交通扰动影响也越大。
如何构建一个高效稳定的交通系统,提高乘客的出行满意度,
以吸引更多乘客选择公共交通出行,对推动绿色出行和社会发展有着重大意义。
随着数据挖掘、机器学习等

本文以南京信息工程大学内的小型公交系统为例,对公交的刷卡数据、线路数据、
天气数据等进行处理,通过大数据分析和机器学习算法对乘客的行为进行挖掘,
优化站点设置和发车策略,提高乘客出行质量,增强公共交通的竞争力。

\subsection{本模板介绍}
鉴于Microsoft Word 编写论文排版工作量大且枯燥,为了实现论文的排版的自动化、规范化,按照《南京信息工程大学本科生毕业论文(设计)撰写排版规范》编写了可用于南京信息工程大学本科生毕业论文的\LaTeX 模板。

模板2021.6版本修订者根据2021年的论文格式要求修订该模板,并使用该模板完成毕业论文工作。
\section{研究相关理论}

\subsection{适合人群}

\section{模型}

\subsection{问题描述}
首先基于统计数据,
然后建立交通网络的模型,根据模型列出对应的VSP问题方程并求解,以便之后优化问题的求解和模拟,
使用SimPy包对不同结果进行模拟。

\subsection{单服务台负指数分布排队系统}
在串行的k个车站中,每个车站的服务时间相互独立,服务时间服从参数为 $k \mu$ 的负指数分布,称总服务时间服从 $k$ 阶爱尔郎分布。
爱尔朗分布族提供更为广泛的模型类,比指数分布有更大的适应性。事实上,当 k = 1 时, 爱尔朗分布化为负指数分布,进一步可推导出对于单个车站
$k=1$ 的情况下到达人数近似服从泊松分布,由于泊松分布的无记忆性\cite{gll},可以更好的描述一个排队系统。
\begin{figure}[htbp!]
    \centering
    \subfigure[k=1的爱尔郎分布\label{fig31:sub1}]{\includegraphics[width=0.4\textwidth]{figs/chap03/elarng.png}}
    \subfigure[泊松分布\label{fig31:sub2}]{\includegraphics[width=0.4\textwidth]{figs/chap03/possion.png}}
    \caption{两种分布对比图}
    \label{dis1}
\end{figure}


前文给出了单服务台负指数分布排队系统(M/M/1/ $\infty $ / $\infty $)的基本方程~\ref{fomula1}~和状态图~\ref{fig21}~,由图图~\ref{fig21}~
可知,状态 $0$ 转移到状态 $1$ 的转移率为 $\lambda P_0$ ,状态 $1$ 转移到状态 $0$ 的转移率为 $\mu P_1$。由排队系统生灭状态的平衡性质可知,对于状态0必须满足如下方程:
$$\lambda P_0 = \mu P_1$$
对于任意的 $n \ge 1$ 的状态,都可以得到式~\ref{fomula1}~中的方程,求解~\ref{fomula1}~得:
$$P_1 = (\lambda / \mu)P_0$$
易证:
$$P_2 = \left(\lambda / \mu \right)^2 P_0$$
$$……$$
$$P_n = \left(\lambda / \mu \right)^n P_0$$
令 $\rho = \frac{\lambda}{\mu} < 1$,由概率的性质
$$\sum_{n=0}^{\infty}p_n = 1$$ 可以得到站台中乘客数为n的概率 $$P_n = (1-\rho)\rho^n, n \le 1, \rho < 1$$
其中 $\rho$ 代表系统的平均服务率,它刻画了服务机构的繁忙程度;所以又称服务机构的利用率。

由此可以得到一个平均到达率为 $\lambda$,平均服务率 $\mu$ 的排队系统的几个主要指标:

(1)乘客到达后不能及时得到服务需要等待的概率(系统服务强度):
$$P_w = \frac{\lambda}{\mu}$$

(2)系统空闲的概率:
$$P_0 = 1-\frac{\lambda}{\mu}$$

(3)系统中平均乘客长度(队长的期望值)
$$L_s = \frac{\lambda}{\mu - \lambda}$$

(4)在队列中等待的平均顾客数(队列长期望值)
$$L_q = L_s - \rho = \frac{\rho \lambda}{\mu - \lambda}$$

(5)在系统中顾客逗留时间的期望值
$$W_s = E[W] = \frac{1}{\mu - \lambda}$$

(6)队列中顾客等待时间的期望值
$$W_q = W_s - \frac{1}{\mu} = \frac{\rho}{\mu - \lambda}$$


解决排队问题首先要根据原始资料作出乘客到达间隔和服务时间的经验分布,根据以往经验,校园公交车站客流呈周期性变化,
选择文德楼北上行线车站实地统计,该车站位于校门、宿舍楼、主教学楼连接处的枢纽位置,客流量较有代表性,采样地点如图~\ref{fig31}~。
\begin{figure}[htbp!]
    \centering
    \includegraphics[width=0.80\textwidth]{figs/chap03/map.png}
    \caption{采样地点}
    \label{fig31}
\end{figure}

以大约两节课的时间(两小时)为一个周期,5分钟为间隔进行采样,所得的一个周期内的客流分布如表~\ref{table_1}~所示。
\begin{table}[htbp!]
    \centering
    \caption{车站乘客到达人数分布表}\label{table_1}
    
    \begin{tabular}{cccc}
    \whline 
    时间段 & 到达人数 & 时间段 & 到达人数 \\ 
    \hline 
    14:00$ \sim $14:05 & 18 &15:00$ \sim $15:05&5\\ 
    14:06$ \sim $14:10 & 7 &15:06$ \sim $15:10&2\\ 
    14:11$ \sim $14:15 & 4 &15:11$ \sim $15:15&3\\ 
    14:16$ \sim $14:20 & 8 &15:16$ \sim $15:20&6\\ 
    14:21$ \sim $14:25 & 1 &15:21$ \sim $15:25&5\\ 
    14:26$ \sim $14:30 & 5 &15:26$ \sim $15:30&67\\ 
    14:31$ \sim $14:35 & 1 &15:31$ \sim $15:35&8\\ 
    14:36$ \sim $14:40 & 3 &15:36$ \sim $15:40&12\\ 
    14:41$ \sim $14:45 & 5 &15:41$ \sim $15:45&8\\ 
    14:46$ \sim $14:50 & 6 &15:46$ \sim $15:50&3\\ 
    14:51$ \sim $14:55 & 2 &15:51$ \sim $15:55&4\\ 
    14:56$ \sim $15:00 & 2 &15:56$ \sim $16:00&2\\ 
    \whline 
    \end{tabular}
    \end{table}
\section{试验和结果}
\subsection{实验环境}
本文实验的训练和测试均搭建于Linux操作系统,版本为Ubuntu20.06 LTS。使用的处
理器为Intel CORE I7,开发工具为 Ananconda3、python3.8、Pycharm。模型的开发基于
Simpy 4.0.1、Pyomo 6.5.0、gurobipy 10.0框架。

\subsection{数据准备}
解决排队问题首先要根据原始资料作出乘客到达间隔和服务时间的经验分布,根据以往经验,校园公交车站客流呈周期性变化,
选择文德楼北上行线车站实地统计,该车站位于校门、宿舍楼、主教学楼连接处的枢纽位置,客流量较有代表性,采样地点如图~\ref{fig31}~。
\\
\begin{figure}[htbp!]
    \centering
    \includegraphics[width=0.7\textwidth]{figs/chap03/map.png}
    \caption{采样地点}
    \label{fig31}
\end{figure}

原始数据记载乘客到达时刻和对应的等待时间,以 $\tau_i$ 表示第i个乘客到达的时刻,以s表示等待时间可以得出相继到达时间 $t_i \left(t_i = \tau_{i+1} - \tau_i\right)$ 和等待时间 $w_i$,
它们的关系如图~\ref{fig32}~。

\begin{figure}[htbp!]
    \center
    \subfigure[$w_i + s_i - t_i > 0$]{\label{queue1}
    \includegraphics[width=0.7\textwidth]{figs/chap03/queue1.png}
    }
    \\
    \subfigure[$w_i + s_i - t_i < 0$]{\label{queue2}
    \includegraphics[width=0.7\textwidth]{figs/chap03/queue2.png}
    }
    \caption{相继到达的间隔时间和排队等待时间关系示意图}\label{fig32}
\end{figure}

由图~\ref{fig32}~可以得到如下关系:

间隔\quad \quad \quad \quad$t_i = \tau_{i+1} - \tau_i$

等待时间 \quad \quad 
$
w_{i+1} = 
\begin{cases}
    w_i + s_i - t_i,\quad w_i + s_i - t_i > 0
    \\
   0, \qquad \qquad \quad w_i + s_i - t_i < 0
\end{cases}
$


% \begin{figure}[htbp!]
%     \center
%     \subfigure[$w_i + s_i - t_i > 0$]{\label{queue1}
%     \includegraphics[width=0.5\textwidth]{figs/chap03/queue1.png}
%     }
%     \\
%     \subfigure[$w_i + s_i - t_i < 0$]{\label{queue2}
%     \includegraphics[width=0.5\textwidth]{figs/chap03/queue2.png}
%     }
%     \caption{相继到达的间隔时间和排队等待时间关系示意图}\label{fig32}
% \end{figure}

由图~\ref{fig32}~可以得到如下关系:

间隔\quad \quad \quad \quad$t_i = \tau_{i+1} - \tau_i$

等待时间 \quad \quad 
$
w_{i+1} = 
\begin{cases}
    w_i + s_i - t_i,\quad w_i + s_i - t_i > 0
    \\
   0, \qquad \qquad \quad w_i + s_i - t_i < 0
\end{cases}
$



以大约两节课的时间(两小时)为一个周期,5分钟为间隔进行采样,将所统计的一个周期内的客流数据
经以上关系整理后所得结果如表~\ref{table_1}~所示。
\begin{table}[htbp!]
    \centering
    \caption{车站乘客到达人数分布表}\label{table_1}
    
    \begin{tabular}{cccc}
    \whline 
    时间段 & 到达人数 & 时间段 & 到达人数 \\ 
    \hline 
    14:00$ \sim $14:05 & 18 &15:00$ \sim $15:05&5\\ 
    14:06$ \sim $14:10 & 7 &15:06$ \sim $15:10&2\\ 
    14:11$ \sim $14:15 & 4 &15:11$ \sim $15:15&3\\ 
    14:16$ \sim $14:20 & 8 &15:16$ \sim $15:20&6\\ 
    14:21$ \sim $14:25 & 1 &15:21$ \sim $15:25&5\\ 
    14:26$ \sim $14:30 & 5 &15:26$ \sim $15:30&67\\ 
    14:31$ \sim $14:35 & 1 &15:31$ \sim $15:35&8\\ 
    14:36$ \sim $14:40 & 3 &15:36$ \sim $15:40&12\\ 
    14:41$ \sim $14:45 & 5 &15:41$ \sim $15:45&8\\ 
    14:46$ \sim $14:50 & 6 &15:46$ \sim $15:50&3\\ 
    14:51$ \sim $14:55 & 2 &15:51$ \sim $15:55&4\\ 
    14:56$ \sim $15:00 & 2 &15:56$ \sim $16:00&2\\ 
    \whline 
    \end{tabular}
    \end{table}

统计同时段内车辆到达时间,可以估算出乘客的平均等待时间,如表~\ref{teble_2}~所示。
\begin{table}[htbp!]
    \centering
    \caption{车站平均等待时间表}\label{teble_2}
    \begin{tabular}{cc}
        \whline
        等待时间(分钟) & 频次 \\
        0 & 23\\
        1 & 31\\
        2 & 27\\
        3 & 25\\
        4 & 33\\
        5 & 19\\
        6 & 22\\
        7 & 19\\
        8 & 8\\\
        >10&0 \\
        \whline
    \end{tabular}
\end{table}

\subsection{评价指标}
使用Gurobi对该算法进行多次迭代计算所得结果如图~\ref{fig44}~
\begin{figure}[htbp!]
    \centering
    \includegraphics[width=0.4\textwidth]{figs/chap04/myplot.png}
    \caption{求解过程迭代图}
    \label{fig44}
\end{figure}

在迭代到一定程度后该算法不再收敛。选择一个可行解,以平均队列长度为评价指标与原有方法比较。

\subsection{实验与分析}

由统计结果可以得到公交站这一排队系统的几个特征:

平均到达率:$$\lambda \approx 1.56$$ 

平均等待时间:4.30(min)

平均服务率:$$\mu = 1/4.30 \approx 0.23$$

系统的服务强度:$$\rho = 6.67 $$

因为服务强度远远大于1,说明系统的服务能力不足以应对到达率,系统会出现排队现象,
排队长度可能会不断增加。如果希望系统稳定运行,需要增加服务能力或降低到达率。通过实际分析,系统在高峰时期更容易产生阻塞,
可以在高峰时期采用大小区间的方案解决这一问题。分别取t=60、120、180、240进行模拟,可以得到原始方案中用户到达服务图:
\begin{figure}[htbp]
    \center
    \subfigure[t=60]{\label{601}
    \includegraphics[width=0.4\textwidth]{figs/chap03/601.png}
    }\subfigure[t=120]{\label{1201}
    \includegraphics[width=0.4\textwidth]{figs/chap03/1201.png}
    }
    \\
    \subfigure[t=180]{\label{1801}
    \includegraphics[width=0.4\textwidth]{figs/chap03/1801.png}
    }\subfigure[t=240]{\label{2401}
    \includegraphics[width=0.4\textwidth]{figs/chap03/2401.png}
    }
    \caption{系统时间-队列-服务图}\label{fig34}
\end{figure}


系统的时间-队列长度图:
\begin{figure}[htbp]
    \center
    \subfigure[t=60]{\label{602}
    \includegraphics[width=0.4\textwidth]{figs/chap03/602.png}
    }\subfigure[t=120]{\label{1202}
    \includegraphics[width=0.4\textwidth]{figs/chap03/1202.png}
    }
    \\
    \subfigure[t=180]{\label{1802}
    \includegraphics[width=0.4\textwidth]{figs/chap03/1802.png}
    }\subfigure[t=240]{\label{2402}
    \includegraphics[width=0.4\textwidth]{figs/chap03/2402.png}
    }
    \caption{系统时间-队列长度图}\label{fig35}
\end{figure}

系统的时间-等待时间图:
\begin{figure}[htbp]
    \center
    \subfigure[t=60]{\label{603}
    \includegraphics[width=0.45\textwidth]{figs/chap03/603.png}
    }\subfigure[t=120]{\label{1203}
    \includegraphics[width=0.45\textwidth]{figs/chap03/1203.png}
    }
    \\
    \subfigure[t=180]{\label{1803}
    \includegraphics[width=0.45\textwidth]{figs/chap03/1803.png}
    }\subfigure[t=240]{\label{2403}
    \includegraphics[width=0.45\textwidth]{figs/chap03/2403.png}
    }
    \caption{系统时间-等待时间图}\label{fig36}
\end{figure}

选择一个最优解,得到如下运行图。
\begin{figure}[htbp!]
    \centering
    \includegraphics[width=0.4\textwidth]{figs/chap04/time.png}
    \caption{优化后的运行图}
    \label{fig45}
\end{figure}

在该实验中,按照3.2中介绍的模型构建方法使用Gurobi搭建模型。
通过迭代计算,在迭代次数超过180后模型开始不收敛。在不考虑大小区间的情况下,所得的可行解可以得到一个较好的方案。

使用表~\ref{table_1}~中的数据,通过图~\ref{fig32}~中的关系重新计算指标,使用大小区间方案处理后可以得到:
$$
\rho = 0.84
$$

对该系统进行模拟结果如下:
\begin{figure}[htbp!]
    \center
    \subfigure[系统时间-队列-服务图]{\label{1}
    \includegraphics[width=0.5\textwidth]{figs/chap04/01.png}
    }
    \\
    \subfigure[系统时间-队列长度图]{\label{2}
    \includegraphics[width=0.5\textwidth]{figs/chap04/02.png}
    }
    \\
    \subfigure[系统时间-等待时间图]{\label{3}
    \includegraphics[width=0.5\textwidth]{figs/chap04/03.png}
    }
    \caption{优化后系统的模拟结果}\label{fig47}
\end{figure}

\section{总结与展望}
本文基于传统的启发式方法,使用一种包含多个约束的目标规划算法,该算法灵感来源于
\cite{CHIERICI200499,2002},对传统的有规则时刻表计算任务添加多个目标约束,
同时由 Gurobi Optimization 公司开发新一代大规模优化器Gurobi,为该类问题的
建模和求解提供了帮助。当实际问题越来越复杂,问题规模越来越庞大的时候,需要一个经过
证明可以信赖的大规模优化工具,为决策提供质量保证。基于3.1中经典的排队理论,基于真实的
统计数据,可以使用离散系统建模包SimPy对车站客流进行模拟。
本文重点讨论了如何通过计算和调整具有给定时间变化的乘客的时刻表,使乘客在车站的总等待时间最小。
建立了一个统一的带有线性约束的二次整数规划模型,来协调每个车站的乘客出发时间。
如图~\ref{fig34}~到图~\ref{fig36}~的模拟结果,通过平均队列长度、平均等待时间等指标,
分析现有系统的缺陷,给出适当的优化方案。通过多次迭代解得出的新的发车表(图~\ref{fig45}~)
在理论上可以有效提高系统的平均服务强度。

同时本文的优化问题具有一定的普遍性,在大数据时代,机器学习、深度学习技术热火朝天的同时,
优化问题再一次进入人们的视野,“十年电气无人问,一朝ML人皆知”,这句话可见现有新研究对优化问题的重视。
随着AI技术的发展,对机器学习底层的最优化理论的重要性已经开始逐渐意识到了。
最优化理论是机器学习算法的核心,包括梯度下降、牛顿法、拟牛顿法等一系列优化算法,
这些算法的效率和收敛性直接影响到机器学习的性能和应用。理解最优化理论的基本原理和方法,
对于开发高效、鲁棒的机器学习算法至关重要。越来越多的人开始关注最优化理论,并且开始进行相关的研究和开发。
最优化以蕴含在机器学习内部,作为参数寻优的调整方向。大可以作为独立模型,在机器学习得到的数据模型的基础上做决策模型,
比如现在的随机优化已有先拟合分布再建立优化的方向。

本文同样还存在许多的不足,由于数据采集条件有限,对于客流分布的描述仅仅基于少量的统计数据得出,
进行计算后给出客流的分布函数,对于乘客出行兴趣、O-D对等数据没有进行详尽的统计。
事实上客流受非常多的因素影响,仅仅用简单的分布函数难以完整的描述出该模式。
实际情况下,车辆间的协同通信让模型具有更高的复杂性。
由于道路网络时变的交通模式和复杂的空间依赖性,交通流预测是一个具有挑战性的时空预测问题。
为了克服该挑战,新的研究将交通网络看为一张图,提出新的深度学习预测模型,
交通图卷积长短时记忆网络(TGC-LSTM)学习交通网络中道路之间的相互作用,并预测网络级的交通状态。
近年来的一些研究\cite{bjtu2020}也表明,传统的LSTM网络在对于融合乘客出行兴趣、O-D对、天气等因素的复杂场景下,
对于客流的预测有着不俗的表现。对于更加复杂的场景,融合时空的双向数据下,基于self-attention的算法\cite{transformer2022}
更是有着优于LSTM的表现。现有的交通流预测方法大部分采用基于区域的态势感知图像或基于站点的图像表示去捕捉交通流的空间动态性,
而态势感知图像和图形表示的结合同样是精准预测的关键。
在对于优化模型的的求解,本文提出的模型在迭代次数超过一定程度后,不具有良好的
收敛性。对于大规模数值优化问题的求解,基于深度学习的方法\cite{nair2021solving}近年来也进入了人们的视野。显然,当前大数据时代,
基于数据驱动的机器学习方法是大势所趋,传统优化方法与新技术的结合受到越来越多的重视。期待在有了更加优质的数据的支持下,
可以使用新的方法对该问题进行更深入的研究。




% \input{body/math.tex}
% \input{body/figs.tex}
% \input{body/table.tex}
% \input{body/code.tex}
% \input{body/ref.tex}
%%% 正文部分

% \section{写在后面(第一版作者)}

% 时间过得真快,从五一动手,到码字码到这里差不多快三天了。这么短的时间,不管模板本身还是说明文档肯定还是不够完善的。但时间所迫,也必须到这了。

% 有人也许行会产生疑问,word不是用着挺好的吗,干嘛要学这个,干嘛要用这个写论文呢?其实要我回答呢,的确是这样的,随便用哪个排版软件用顺手了就好了,没人强迫你做什么,关键在于自己是怎么想。

% 去年笔者在写学年论文时,就“吃了亏”,先是用\LaTeX 写的,生成的pdf格式的文档,但是最后学院不认,说必须用word版的,无奈后来又用word重排了一遍。(所以这里插一句,如果真有哪位朋友想用这个模板,请“严重”地考虑这个“严重”的后果,弄不好到最后,只能用它排个打印版玩玩,电子版还得word 去。)

% 而对笔者来说,无所谓,不是天空经常会飘来五个字儿,叫“这都不是事儿”嘛,人生本就是向死之生,要是总走直线,太快到终点了怎么办?所以人生的要义就在于走“弯路”,走得越“弯”走得越长嘛。

% \input{body/2nd.tex}
% \input{body/2021.6.tex}
% \input{body/2022.tex}
% \input{body/2022.4.tex}

%%% 参考文献
\input{body/bib.tex}
%%% 参考文献


%% 致谢
\thanking
{
行文至此,大学的学习生活似乎也将告一段落了。
首先,谨向我的指导老师马杰良老师表示衷心的感谢。马老师不仅无私地向我们传授知识,还关心体贴学生,
善于因材施教。

感谢所有任课老师,感谢我的辅导员和班主任,感谢他们的谆谆教诲
和鼓励帮助。

感谢南信大的全体后勤人员,因为她们的存在学生们才能在多变的环境下安
心的学习和生活。

感谢过去几年那个刻苦钻研,遇到困难不退缩的自己,从一个懵懂的高中生成长成一个能独当一面的"coder"

一路花开,何德何能,所遇之人皆携手相助,不偏不倚,与君相识,三生有幸。
虽陨首结草不能报之万一,借文聊表谢忱!
}
%% 致谢

\end{document}